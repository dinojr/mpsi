% Author: Julien Cubizolles
% Public domain

\protected\def\Referentiel#1{\mathcal #1}
\protected\def\ReferentielBar{\Referentiel{R}^*}

\protected\def\Energie{\mathcal{E}}
\protected\def\EnergieCinetique#1#2{\mathcal{E}_{c,\Referentiel{#2}}(#1)}
\protected\def\EnergieCinetiqueSys#1#2{\mathcal{E}_{c,\Referentiel{#2}}(\mathcal
  #1)}
\protected\def\EnergiePotentielle{\mathcal{E}_p}
\protected\def\EnergiePotentielleEff{\mathcal{E}_{p,\mathrm{eff}}}
\protected\def\EnergieMecanique#1#2{\mathcal{E}_{m,\Referentiel{#2}}(#1)}
\protected\def\EnergieMecaniqueSys#1#2{\mathcal{E}_{m,\Referentiel{#2}}(\mathcal
  #1)}

\protected\def\Dimension#1{\mathsf{#1}}

\protected\def\Puissance{%
  \mathcal{P}}
\protected\def\PuissanceR#1#2{%
  \Puissance^{#1}_{\text{reçu par #2}}}
\protected\def\PuissanceF#1#2{%
  \Puissance^{#1}_{\text{fournie par #2}}}
\protected\def\PuissanceInt{%
  \Puissance_{\mathrm{int}}}
\protected\def\PuissanceExt{%
  \Puissance_{\mathrm{ext}}}

\def\EpotEffective{\mathcal{E}_\text{eff}}
\def\Epotelec{\Epot[\text{élec}]}
\def\Eelec{\Energie_\text{élec}}
\def\Emag{\Energie_\text{mag}}

\protected\def\Epot{\@ifnextchar[{\Epot@ii}{\Epot@i}}
\protected\def\Epot@i{\Energie_{\text{pot}}}
\protected\def\Epot@ii[#1]{\Energie_{\text{pot},#1}}

\protected\def\Vitesse{%
\@ifnextchar*{\Vitesse@i}{\Vitesse@ii}}
\def\Vitesse@ii#1#2{%
  \vv{v_{\Referentiel{#2}}}(#1)}
\def\Vitesse@i*#1{%
  \vv{v}(#1)}
\protected\def\QteMvtSys#1#2{%
  \vv{p_{\Referentiel{#2}}}(\mathcal S)}
\protected\def\QteMvt{%
  \@ifnextchar*{\QteMvt@i}{\QteMvt@ii}}
\def\QteMvt@ii#1#2{%
  \vv{p_{\Referentiel{#2}}}(#1)}
\def\QteMvt@i*#1{%
  \vv{p}(#1)}
\protected\def\Acceleration{%
  \@ifnextchar*{\Acceleration@i}{\Acceleration@ii}}
\def\Acceleration@ii#1#2{%
  \vv{a_{\Referentiel{#2}}}(#1)}
\def\Acceleration@i*#1{%
  \vv{a}(#1)}
\protected\def\AccelerationEnt#1#2#3{%
  \vv{a_{\mathrm{e},\Referentiel{#2}/\Referentiel{#3}}}(#1)}
\protected\def\AccelerationCoriolis#1#2#3{%
  \vv{a_{\mathrm{c},\Referentiel{#2}/\Referentiel{#3}}}(#1)}

\protected\def\MomentCinetique#1#2#3{%
  \vv{\sigma_{\Referentiel{#3}/#2}}(#1)}
\protected\def\MomentCinetiqueAxe#1#2#3{%
  \sigma_{\Referentiel{#3}/#2}(#1)}
\protected\def\MomentCinetiqueSys#1#2#3{%
  \vv{\sigma_{\Referentiel{#3}/#2}}(\mathcal #1)}
\protected\def\MomentCinetiqueSysAxe#1#2#3{%
  \sigma_{\Referentiel{#3}/#2}(\mathcal #1)}

\protected\def\Force#1{\vv{#1}}
\protected\def\ForceEnt{\Force{f_{\mathrm{i,e}}}}
\protected\def\ForceCoriolis{\Force{f_{\mathrm{i,c}}}}

\protected\def\Moment{%
  \@ifnextchar*{\Moment@i}{\Moment@ii}}
\def\Moment@i*#1#2{%
  \vv{\mathcal{M}_{/#1}(#2)}}
\def\Moment@ii#1#2{%
  \vv{\mathcal{M}_{/#1}}(\Force{#2})}
\protected\def\MomentAxe{%
  \@ifnextchar*{\MomentAxe@i}{\MomentAxe@ii}}
\def\MomentAxe@i*#1#2{%
  \mathcal{M}_{/#1}(#2)}
\def\MomentAxe@ii#1#2{%
  \mathcal{M}_{/#1}(\Force{#2})}

\RequirePackage{siunitx}
\sisetup{%
  locale = FR,
  inter-unit-separator = \ensuremath{{} \cdot {}},
  output-decimal-marker = {,},
  exponent-product = \ensuremath{{} \cdot {}},
  group-four-digits = false,
  per-mode = reciprocal,
  separate-uncertainty = false,
  multi-part-units = single}
\DeclareSIUnit[number-unit-product = {}]\dioptrie{\ensuremath{\deltaup}}
\DeclareSIUnit[number-unit-product = {}]\decade{\ensuremath{\mathrm{d\acute{e}cade}}}
\DeclareSIUnit[number-unit-product = {}]\torr{Torr}
\DeclareSIUnit\gauss{G}
\DeclareSIUnit{\EUR}{\text{\euro}}

\def\valeur{%
        \@ifnextchar[\valeur@i\valeur@ii}
\def\valeur@i[#1]#2#3{%
        \SI[scientific-notation=true,%
            round-mode=figures,%
            round-precision=#1]{#2}{#3}}
\def\valeur@ii#1#2{%
        \SI{#1}{#2}}

\RequirePackage{esvect,esdiff}

\newcommand{\Base}{\mathcal{B}}
\newcommand{\Solide}{\mathcal{S}}
\newcommand{\Courbe}{\mathcal{C}}
\newcommand{\syst}{\mathcal{S}}
\protected\def\Ensemble{\{M_i\}_{i=1..N}}
\protected\def\EnsembleMasses{\{m_i\}_{i=1..N}}
\newcommand{\VitAng}{\omega}
\newcommand{\Axe}{\Delta}
\newcommand{\Defemph}[1]{\emph{#1}}
\newcommand{\Textemph}[1]{\emph{#1}}
\usepackage{xspace}
\newcommand{\cad}{c'est-à-dire\xspace}
\newcommand{\RefR}{\Referentiel{R}}
\newcommand{\rond}{\circ}

\newcommand{\Ecin}{{\Energie_{\text{c}}}}
\newcommand{\EcinM}{{\Energie_{\text{cM}}}}
\newcommand{\Emec}{{\Energie_{\text{m}}}}
\newcommand{\EmecM}{{\Energie_{\text{mM}}}}
\newcommand{\Uint}{{U}}

\newcommand{\Circuit}{\mathcal{C}}
\newcommand{\Bobine}{\mathcal{B}}
\newcommand{\cyc}{\ensuremath{\circlearrowleft}}

\def\ForceLorentz{\Force{F_\mathcal {L}}}
\def\Potentiel{V}
\protected\def\Laplace{%
  \@ifnextchar*{\Laplace@i}{\Laplace@ii}}
\def\Laplace@i*{%
   F_{\mathcal{L}a}}
\def\Laplace@ii{%
   \Force{F_{\mathcal{L}a}}
}

\protected\def\Lo{%
   \@ifnextchar*{\Lo@i}{\Lo@ii}}
\def\Lo@i*#1#2{\MomentCinetique{#1}{#2}{\empty}}
\def\Lo@ii#1#2#3{\MomentCinetique{#1}{#2}{#3}}

\protected\def\LD{%
   \@ifnextchar*{\LD@i}{\LD@ii}}
\def\LD@i*#1#2{\MomentCinetiqueAxe{#1}{#2}{\empty}}
\def\LD@ii#1#2#3{\MomentCinetiqueAxe{#1}{#2}{#3}}

\protected\def\MDext#1#2{%
    \mathcal{M}_{\text{ext}/#1}(#2)
}

\protected\def\MR#1{\vv{\mathcal{M}_{/#1}}}
\protected\def\MRext#1{\vv{\mathcal{M}_{\text{ext}/#1}}}
\protected\def\MRAxe#1{\mathcal{M}_{/#1}}
\protected\def\MRextAxe#1{\mathcal{M}_{\text{ext}/#1}}

\protected\def\Couple{%
   \@ifnextchar*{\Couple@i}{\Couple@ii}}
\def\Couple@i*{\vv{\mathcal{C}}}
\def\Couple@ii#1{\mathcal{C}_{/#1}}

\protected\def\MomentInertie#1{J_{#1}}

\def\Champ#1{\vv{#1}}

\def\Lc{\sigma_c}

\RequirePackage{esdiff}
\let\Deriv\diff
\let\Derivp\diffp
\protected\def\DiffOp{\mathrm{d}}
\protected\def\Diff#1{\mathop{}\!\DiffOp #1}

\protected\def\DerivRef{%
  \@ifnextchar*{\DerivRef@i}{\DerivRef@ii}}
\def\DerivRef@i*#1#2#3{%
  \Deriv*{#1}{#2}{\Referentiel{#3}}}
\def\DerivRef@ii#1#2{%
  \Deriv*{#1}{t}{\Referentiel{#2}}}

\protected\def\DerivRefDeux{%
  \@ifnextchar*{\DerivRefDeux@i}{\DerivRefDeux@ii}}
\def\DerivRefDeux@i*#1#2#3{%
  \Deriv*[2]{#1}{{#2}{\Referentiel{#3}}}}
\def\DerivRefDeux@ii#1#2{%
  \Deriv*[2]{#1}{t}{\Referentiel{#2}}}


\let\Delem\Diff

\protected\def\Puiss{%
  \@ifnextchar*{\Puiss@i}{\Puiss@ii}}
\def\Puiss@i*#1#2{%
  \mathcal{P}_{#2}(#1)}
\def\Puiss@ii#1{%
  \mathcal{P}(#1)}
\protected\def\Trajet#1#2#3{
  #1 \xrightarrow[#3]{}#2}

% \protected\def\Trav{%
%   \@ifnextchar*{\Trav@i}{\Trav@ii}}
% \def\Trav@ii{\@ifnextchar[{\Trav@j}{\Trav@jj}}
% \def\Trav@j[#1]#2{W_#1\left(#2\right)}
% \def\Trav@jj#1{W\left(#1\right)}
% \def\Trav@i*{\@ifnextchar[{\Trav@k}{\Trav@kk}}
% \def\Trav@k[#1]#2#3{\underset{#3}{W_#1}\left(#2\right)}
% \def\Trav@kk#1#2{\underset{#2}{W}\left(#1\right)}

\protected\def\Trav{%
  \@ifnextchar*{\Trav@i}{\Trav@ii}}
\def\Trav@ii{\@ifnextchar[{\Trav@j}{\Trav@jj}}
\def\Trav@j[#1]#2{W_#1(#2)}
\def\Trav@jj#1{W(#1)}
\def\Trav@i*{\@ifnextchar[{\Trav@k}{\Trav@kk}}
\def\Trav@k[#1]#2#3{\underset{#3}{W_#1}(#2)}
\def\Trav@kk#1#2{\underset{#2}{W}(#1)}

\def\bouton#1{\textbf{\texttt{#1}}}