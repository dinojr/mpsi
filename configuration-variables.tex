%%%%%%%%%%%%%%%%%%%%%%%%%%%%%%%
%%%%  Variables, fonctions 
%%%%%%%%%%%%%%%%%%%%%%%%%%%%%%%
\newcommand\@classe{MPSI2}
\newcommand\classe[1]{\renewcommand\@classe{#1}}
\newcommand\@lycee{Louis le Grand}
\newcommand\lycee[1]{\renewcommand\@lycee{#1}}
\newcommand\@auteur{Julien Cubizolles}
\newcommand\auteur[1]{\renewcommand\@auteur{#1}}
\newcommand\@titre{Mettre un titre}
\newcommand\titre[1]{\renewcommand\@titre{#1}}
\RequirePackage{url}
\newcommand\@licence{sous licence \tiny\href{http://creativecommons.org/licenses/by-nc-nd/2.0/fr/}{\url{http://creativecommons.org/licenses/by-nc-nd/2.0/fr/}}.}
\newcommand\licence[1]{\renewcommand\@licence{#1}}
\newcommand\@anneesco{2021--2022}
\newcommand\anneesco[1]{\renewcommand\@anneesco{#1}}

\RequirePackage{calc}
\newcounter{hours}
\newcounter{minutes}
\newcommand{\printtime}{%
  \setcounter{hours}{\time/60}%
  \setcounter{minutes}{\time-(\value{hours}*60)}
  \thehours h \theminutes min}

\newcommand{\mois}{%
    \ifcase\month
        \or janvier%
        \or février%
        \or mars%
        \or avril%
        \or mai%
        \or juin%
        \or juillet%
        \or août%
        \or septembre%
        \or octobre%
        \or novembre%
        \or décembre%
    \fi
}
\newcommand{\aujourdhui}{\the\day~\mois~\the\year}
%\newcommand{\timestamp}{\aujourdhui, à \printtime}
\newcommand{\timestamp}{\aujourdhui}
\newcommand{\@jour}{\aujourdhui}
\newcommand{\jour}[1]{\renewcommand{\@jour}{#1}}
\newcommand{\@pourle}{\aujourdhui}
\newcommand{\pourle}[1]{\renewcommand{\@pourle}{#1}}
\newcommand{\makepourle}{Pour le \@pourle}

\renewcommand{\thefootnote}{\roman{footnote}}

\def\maketitle{%
  \par
  \begingroup
  \if@twocolumn
  \twocolumn[\@maketitle]
  \else \newpage
  \global\@topnum\z@ 
  \@maketitle
  \fi
  \endgroup
  \setcounter{footnote}{0}
  \let\maketitle\relax%
  \let\@maketitle\relax%
}

\def\@maketitle{%
  \newpage
  \null
  \vskip 2em 
  \begin{center}
    {\LARGE \textsf{\textbf{ \@title}} \par} % Changer ici pour le style de fonte
  \end{center}
  \par
  \vskip 1.5em
}
