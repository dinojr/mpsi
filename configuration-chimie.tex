% Author: Julien Cubizolles
% Public domain

\RequirePackage[version=3]{mhchem}

\RequirePackage{chemfig}
\newcommand\setpolymerdelim[2]{\def\delimleft{#1}\def\delimright{#2}}

\def\makebraces(#1,#2)#3#4#5{%
  \edef\delimhalfdim{\the\dimexpr(#1+#2)/2}%
  \edef\delimvshift{\the\dimexpr(#1-#2)/2}%
  \chemmove{
    \node[at=(#4),yshift=(\delimvshift)]
      {$
       \left\delimleft
         \vrule height\delimhalfdim depth\delimhalfdim width0pt
       \right.
      $};
    \node[at=(#5),yshift=(\delimvshift)]
      {$
        \left.
          \vrule height\delimhalfdim depth\delimhalfdim width0pt
        \right\delimright_{\rlap{#3}}
      $};
  }%
}

% À utiliser comme suit:
% \setpolymerdelim()
% \chemfig{
%   CH_2=[:-30]CH-[:30]C(=[:90]O)-[:-30]O
%     -[@{op,.75}]CH_2CH_2-[@{cl,.25}]
%     O-[:30]C(=[:90]O)-[:-30]CH=[:30]CH_2
% }
% \makebraces(5pt,5pt){$\!\!n$}{op}{cl}
%\makebraces(<dim up>,<dim down>){<subscript>}{<opening node name>}{<closing node name>}

\RequirePackage{chemformula}
