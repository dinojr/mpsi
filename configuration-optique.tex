% Author: Christophe Jorssen
% Public domain
\usetikzlibrary{decorations.markings}
\pgfdeclaredecoration{border alternate}{tick}{%
  \state{tick}[switch if less than=+\pgfdecorationsegmentlength to final,
               width=+\pgfdecorationsegmentlength]{%
    \pgfpathmoveto{\pgfpointorigin}
    \pgfpathlineto{%
      \pgfpointpolar{%
        \pgfdecorationsegmentangle}{+\pgfdecorationsegmentamplitude}}}
  \state{final}{%
    \pgfpathmoveto{\pgfpointorigin}
    \pgfpathlineto{%
      \pgfpointpolar{%
        \pgfdecorationsegmentangle}{+\pgfdecorationsegmentamplitude}}
    \pgfpathmoveto{\pgfpointdecoratedpathlast}}
}

\tikzset{%
  LRnoarrow/.style = {thick,Couleur1},
  LR/.style 2 args = {%
    decoration={markings,mark=at position #2 with {\arrow[fill opacity=1]{#1};}},
    postaction={decorate},
    LRnoarrow},
  VirtualLR/.style = {LRnoarrow,dashed},
  LR>/.style = {LR={>}{#1}},
  LR>/.default = {0.5},
  LR>>/.style = {LR={>>}{#1}},
  LR>>/.default = {0.55},
  LR>>>/.style = {LR={>>>}{#1}},
  LR>>>/.default = {0.6},
  LR>>>>/.style = {LR={>>>>}{#1}},
  LR>>>>/.default = {0.65},
  LR>>>>>/.style = {LR={>>>>>}{#1}},
  LR>>>>>/.default = {0.7},
  LR>>>>>>/.style = {LR={>>>>>>}{#1}},
  LR>>>>>>/.default = {0.75},
  ConvergingLens/.style = {ultra thick,<->},
  DivergingLens/.style = {ultra thick,>-<},
  Interface/.style = {ultra thick},
  OpticalAxis/.style = {very thick,->},
  Object/.style = {very thick,->},
  VirtualObject/.style = {very thick,->,dashed},
  Medium/.style = {fill=gray,nearly transparent},
  PlaneMirror/.style = {%
    ultra thick,
    postaction = {%
      decoration = {%
        border alternate,
        segment length=.25cm,
        amplitude=.25cm,
        angle=-135},
      decorate,
      arrows=-,
      thick,
      draw}},
  ConvergingMirror/.style = {%
    ultra thick,
    arrows = half angle 90 reversed-other half angle 90 reversed,
    postaction = {%
      decoration = {%
        border alternate,
        segment length=.25cm,
        amplitude=.25cm,
        angle=-135},
      decorate,
      arrows = -,
      thick,
      draw}},
  DivergingMirror/.style = {%
    ultra thick,
    arrows = other half angle 90 reversed-half angle 90 reversed,
    postaction = {%
      decoration = {%
        border alternate,
        segment length=.25cm,
        amplitude=.25cm,
        angle=-135},
      decorate,
      arrows=-,
      thick,
      draw}},
}

% New library (under development)
\tikzset{%
  optics/.is family,
  optics/.cd,
  % Thin centered optical system
  optical system/.is family,
  optical system/.cd,
  image focal length/.initial = 2.5cm,
  object focal length/.initial = -2.5cm,
  upper height/.initial = 1.25cm,
  lower height/.initial = -1.25cm,
}

\tikzset{%
  % Object
  optics/.cd,
  object/.is family,
  object/.cd,
  is object for/.initial = a,
  height/.initial = 1cm,
}

\pgfdeclareshape{thin centered optical system}{%
  \savedmacro\upperheight{%
    \edef\upperheight{%
      \pgfkeysvalueof{%
        /tikz/optics/optical system/upper height}}}
  \savedmacro\lowerheight{%
    \edef\lowerheight{%
      \pgfkeysvalueof{%
        /tikz/optics/optical system/lower height}}}
  \savedmacro\imagefocallength{%
    \edef\imagefocallength{%
      \pgfkeysvalueof{%
        /tikz/optics/optical system/image focal length}}}
  \savedmacro\objectfocallength{%
    \edef\objectfocallength{%
      \pgfkeysvalueof{%
        /tikz/optics/optical system/object focal length}}}
  % Center
  \savedanchor{\centerpoint}{\pgfpointorigin}
  \anchor{center}{\centerpoint}
  % Top
  \savedanchor{\top}{\pgfpoint{0pt}{\upperheight}}
  \anchor{top}{\top}
  % Bottom
  \savedanchor{\bottom}{\pgfpoint{0pt}{\lowerheight}}
  \anchor{bottom}{\bottom}
  % Principal image focus
  \savedanchor{\principalimagefocus}{\pgfpoint{\imagefocallength}{0pt}}
  \anchor{principal image focus}{\principalimagefocus}
  % Image focal plane top
  \savedanchor{\imagefocalplanetop}{\pgfpoint{\imagefocallength}{\upperheight}}
  \anchor{image focal plane top}{\imagefocalplanetop}
  % Image focal plane bottom
  \savedanchor{\imagefocalplanebottom}{%
    \pgfpoint{\imagefocallength}{\lowerheight}}
  \anchor{image focal plane bottom}{\imagefocalplanebottom}
  % Principal object focus
  \savedanchor{\principalobjectfocus}{\pgfpoint{\objectfocallength}{0pt}}
  \anchor{principal object focus}{\principalobjectfocus}
  % Object focal plane top
  \savedanchor{\objectfocalplanetop}{%
    \pgfpoint{\objectfocallength}{\upperheight}}
  \anchor{object focal plane top}{\objectfocalplanetop}
  % Object focal plane bottom
  \savedanchor{\objectfocalplanebottom}{%
    \pgfpoint{\objectfocallength}{\lowerheight}}
  \anchor{object focal plane bottom}{\objectfocalplanebottom}
  %
  \backgroundpath{%
    \pgfpathmoveto{\bottom}
    \pgfpathlineto{\top}
    \pgfusepath{stroke}
  }
}

\pgfdeclareshape{converging lens}{%
  \savedmacro\upperheight{%
    \edef\upperheight{%
      \pgfkeysvalueof{%
        /tikz/optics/optical system/upper height}}}
  \savedmacro\lowerheight{%
    \edef\lowerheight{%
      \pgfkeysvalueof{%
        /tikz/optics/optical system/lower height}}}
  \savedmacro\imagefocallength{%
    \edef\imagefocallength{%
      \pgfkeysvalueof{%
        /tikz/optics/optical system/image focal length}}}
  \savedmacro\objectfocallength{%
    \edef\objectfocallength{-(\pgfkeysvalueof{%
        /tikz/optics/optical system/image focal length})}}
  %
  \inheritsavedanchors[from=thin centered optical system]
  \inheritanchor[from=thin centered optical system]{center}
  \inheritanchor[from=thin centered optical system]{top}
  \inheritanchor[from=thin centered optical system]{bottom}
  \inheritanchor[from=thin centered optical system]{principal image focus}
  \inheritanchor[from=thin centered optical system]{image focal plane top}
  \inheritanchor[from=thin centered optical system]{image focal plane bottom}
  \inheritanchor[from=thin centered optical system]{principal object focus}
  \inheritanchor[from=thin centered optical system]{object focal plane top}
  \inheritanchor[from=thin centered optical system]{object focal plane bottom}
  %
  \backgroundpath{%
    \pgfsetarrows{stealth-stealth}
    \pgfsetlinewidth{2pt}
    \pgfpathmoveto{\bottom}
    \pgfpathlineto{\top}
  }
}

\tikzset{%
  every converging lens node/.style = {%
    /tikz/optics/optical system/object focal length =
    -(\pgfkeysvalueof{%
      /tikz/optics/optical system/image focal length})}}

\pgfdeclareshape{object}{%
  \savedmacro\height{%
    \edef\height{\pgfkeysvalueof{/tikz/optics/object/height}}}
  \savedmacro\isobjectfor{%
    \edef\isobjectfor{%
      \pgfkeysvalueof{/tikz/optics/object/is object for}}}
  \savedmacro\imagefocallength{%
    \begingroup
      \csname pgf@sh@ma@\isobjectfor\endcsname
      \edef\pgf@temp{%
        \endgroup
        \def\noexpand\imagefocallength{\imagefocallength}}%
      \pgf@temp}
  % Bottom
  \savedanchor{\centerpoint}{\pgfpointorigin}
  \anchor{center}{\centerpoint}
  \anchor{bottom}{\centerpoint}
  % Top
  \savedanchor{\top}{\pgfpoint{0pt}{\height}}
  \anchor{top}{\top}
  % Optical system center
  \savedanchor{\opticalsystemcenter}{%
    \pgfpointanchor{\isobjectfor}{center}}
  \anchor{optical system center}{\opticalsystemcenter}
  % Top on optical system
  \savedanchor{\toponopticalsystem}{%
    % This allows to define \savedanchors in terms of other saved anchors.
    \pgf@sh@savedpoints
    \pgfpointdiff{\top}{\opticalsystemcenter}
    \pgf@y=\height}
  \anchor{top on optical system}{\toponopticalsystem}
  % Image top
  \savedanchor{\imagetop}{%
    % This allows to define \savedanchors in terms of other saved anchors.
    \pgf@sh@savedpoints
    \pgfpointintersectionoflines{%
      \toponopticalsystem}{%
      \pgfpointanchor{\isobjectfor}{principal image focus}}{%
      \top}{%
      \pgfpointanchor{\isobjectfor}{center}}}
  \anchor{image top}{\imagetop}
  % Image bottom
  \savedanchor{\imagebottom}{%
    % This allows to define \savedanchors in terms of other saved anchors.
    \pgf@sh@savedpoints
    \imagetop
    \pgf@y=0pt
  }
  \anchor{image bottom}{\imagebottom}
  % Image top on optical system
  \savedanchor{\imagetoponopticalsystem}{%
    % This allows to define \savedanchors in terms of other saved anchors.
    \pgf@sh@savedpoints
    \pgfpointanchor{\isobjectfor}{center}
    \pgf@xa=\the\pgf@x
    \imagetop
    \pgf@x=\the\pgf@xa}
  \anchor{image top on optical system}{\imagetoponopticalsystem}
  %
  \savedmacro\isimagevirtual{%
    \begingroup
    \pgf@sh@savedpoints
    \opticalsystemcenter
    \pgf@xa=-\the\pgf@x
    \imagetop
    \advance\pgf@xa by\pgf@x
    \edef\pgf@temp{%
      \endgroup
      \ifdim\pgf@xa<0pt
        \def\noexpand\isimagevirtual{1}%
      \else
        \def\noexpand\isimagevirtual{0}%
      \fi}
    \pgf@temp}
  %
  \savedmacro\isobjectvirtual{%
    \begingroup
    \pgf@sh@savedpoints
    \opticalsystemcenter
    \pgf@xa=-\the\pgf@x
    \top
    \advance\pgf@xa by\pgf@x
    \edef\pgf@temp{%
      \endgroup
      \ifdim\pgf@xa<0pt
        \def\noexpand\isobjectvirtual{0}%
      \else
        \def\noexpand\isobjectvirtual{1}%
      \fi}
    \pgf@temp}
  %
  \savedmacro\imageheight{%
    \begingroup
    \pgf@sh@savedpoints
    \imagetop
    \pgf@ya=\the\pgf@y
    \imagebottom
    \advance\pgf@ya by-\pgf@y
    \edef\pgf@temp{%
      \endgroup
      \def\noexpand\imageheight{\the\pgf@ya}}
    \pgf@temp}
  %
  \backgroundpath{%
    \expandafter\xdef\csname pgf@sh@optics@isobjectfor@\pgf@node@name\endcsname{%
      \isobjectfor}%
    \expandafter\xdef\csname pgf@sh@optics@isimagevirtual@\pgf@node@name\endcsname{%
      \isimagevirtual}%
    \expandafter\xdef\csname pgf@sh@optics@isobjectvirtual@\pgf@node@name\endcsname{%
      \isobjectvirtual}%
    \expandafter\xdef\csname pgf@sh@optics@imageheight@\pgf@node@name\endcsname{%
      \imageheight}%
    \pgfsetarrows{-stealth}
    \pgfsetlinewidth{1pt}
    \pgfpathmoveto{\centerpoint}
    \pgfpathlineto{\top}
  }
}

\tikzset{%
  LR through center/.style args = {#1 with #2}{%
    insert path = {%
      \pgfextra{%
        \pgfinterruptpath
          \draw[LR#2] (#1.top) -- (#1.optical system center);
          \draw[LR#2] (#1.optical system center) -- (#1.image top);
        \endpgfinterruptpath
      }
    }
  },
  LR through image focus/.style args = {#1 with #2}{%
    % Il faut distinguer les cas ou B' est avant/après F' pour savoir
    % jusqu'ou on va.
    insert path = {%
      \pgfextra{%
        \pgfinterruptpath
          \expandafter\ifnum\csname pgf@sh@optics@isimagevirtual@#1\endcsname>0\relax
          % The image is virtual
            \expandafter\ifnum\csname pgf@sh@optics@isobjectvirtual@#1\endcsname>0\relax
            % The object is virtual
              \draw[LR#2] (#1.top) -- (#1.optical system center);
              \draw[LR#2] (#1.optical system center) -- (#1.image top);
            \else
            % The object is real
              \draw[LR#2] (#1.top) -- (#1.optical system center);
              \draw[LR#2] (#1.optical system center) -- (#1.image top);
            \fi
          \else
          % The image is real
            \expandafter\ifnum\csname pgf@sh@optics@isobjectvirtual@#1\endcsname>0\relax
            % The object is virtual
              \draw[LR#2] (#1.top) -- (#1.optical system center);
              \draw[LR#2] (#1.optical system center) -- (#1.image top);
            \else
            % The object is real
              \draw[LR#2] (#1.top) -- (#1.top on optical system);
              \draw[LR#2] (#1.top on optical system) -- 
                          (\csname
                          pgf@sh@optics@isobjectfor@#1\endcsname.principal
                          image focus);
            \fi
          \fi
        \endpgfinterruptpath
      }
    }
  },
}



% Local Variables:
% coding: utf-8
% End:
