% Author: Christophe Jorssen
% Public domain
\makeatletter
\usetikzlibrary{circuits.ee.IEC}
\newlength\tikzDipoleWidth
\setlength{\tikzDipoleWidth}{2cm}
% \def\tikzDipoleWidth{2cm}

\tikzset{%
  voltage info/.is family,
  voltage info/vertical offset/.initial = .5,
  voltage info/arrow direction/.initial = ->,
  voltage info/horizontal offset/.initial = 2,
  voltage info/label position/.initial = above,
  voltage info/label rotate/.initial = 0,
  voltage info/label/.initial = {}
}

\tikzset{%
  voltage arrow/.style = {%
    /utils/exec = {%
      \pgfsetarrowoptions{direction ee}{%
        .4*\the\tikzcircuitssizeunit+.3*\the\pgflinewidth}},
    > = direction ee},
  voltage info sloped/.style = {% 
    append after command = {%
      \bgroup
        [voltage info/.cd,#1]
        [current point is local=true]
        [every voltage info/.try]
        [shift={($(\tikzlastnode.north)+(0,\pgfkeysvalueof{/tikz/voltage info/vertical offset}\tikzcircuitssizeunit)$)}]
        [voltage arrow,\pgfkeysvalueof{/tikz/voltage info/arrow direction}]
        (-\pgfkeysvalueof{/tikz/voltage info/horizontal offset}\tikzcircuitssizeunit,0)
        edge[line to] 
        node[\pgfkeysvalueof{/tikz/voltage info/label
          position},transform shape,auto,rotate = \pgfkeysvalueof{/tikz/voltage
          info/label rotate}] {\pgfkeysvalueof{/tikz/voltage info/label}}
        (\pgfkeysvalueof{/tikz/voltage info/horizontal offset}\tikzcircuitssizeunit,0)
      \egroup
    }
  },
  voltage info' sloped/.style = {% 
    append after command = {%
      \bgroup
        [voltage info/.cd,label position = below,#1]
        [current point is local=true]
        [every voltage info/.try]
        [shift={($(\tikzlastnode.south)+(0,-\pgfkeysvalueof{/tikz/voltage info/vertical offset}\tikzcircuitssizeunit)$)}]
        [voltage arrow,\pgfkeysvalueof{/tikz/voltage info/arrow direction}]
        (-\pgfkeysvalueof{/tikz/voltage info/horizontal offset}\tikzcircuitssizeunit,0)
        edge[line to] 
        node[\pgfkeysvalueof{/tikz/voltage info/label
          position},auto,swap,transform shape,rotate = \pgfkeysvalueof{/tikz/voltage
          info/label rotate}] {\pgfkeysvalueof{/tikz/voltage info/label}}
        (\pgfkeysvalueof{/tikz/voltage info/horizontal offset}\tikzcircuitssizeunit,0)
      \egroup
    }
  },
  voltage info/.style = {% 
    append after command = {%
      \bgroup
        [voltage info/.cd,#1]
        [current point is local=true]
        [every voltage info/.try]
        [shift={($(\tikzlastnode.north)+(0,\pgfkeysvalueof{/tikz/voltage info/vertical offset}\tikzcircuitssizeunit)$)}]
        [voltage arrow,\pgfkeysvalueof{/tikz/voltage info/arrow direction}]
        (-\pgfkeysvalueof{/tikz/voltage info/horizontal offset}\tikzcircuitssizeunit,0)
        edge[line to] 
        node[\pgfkeysvalueof{/tikz/voltage info/label
          position},auto,rotate = \pgfkeysvalueof{/tikz/voltage
          info/label rotate}] {\pgfkeysvalueof{/tikz/voltage info/label}}
        (\pgfkeysvalueof{/tikz/voltage info/horizontal offset}\tikzcircuitssizeunit,0)
      \egroup
    }
  },
  voltage info'/.style = {% 
    append after command = {%
      \bgroup
        [voltage info/.cd,label position = below,#1]
        [current point is local=true]
        [every voltage info/.try]
        [shift={($(\tikzlastnode.south)+(0,-\pgfkeysvalueof{/tikz/voltage info/vertical offset}\tikzcircuitssizeunit)$)}]
        [voltage arrow,\pgfkeysvalueof{/tikz/voltage info/arrow direction}]
        (-\pgfkeysvalueof{/tikz/voltage info/horizontal offset}\tikzcircuitssizeunit,0)
        edge[line to] 
        node[\pgfkeysvalueof{/tikz/voltage info/label
          position},auto,swap,rotate = \pgfkeysvalueof{/tikz/voltage
          info/label rotate}] {\pgfkeysvalueof{/tikz/voltage info/label}}
        (\pgfkeysvalueof{/tikz/voltage info/horizontal offset}\tikzcircuitssizeunit,0)
      \egroup
    }
  }
}

\pgfdeclaredecoration{single line}{initial}{%
  \state{initial}[width=\pgfdecoratedpathlength-1sp]{%
    \pgfpathmoveto{\pgfpointorigin}}
  \state{final}{\pgfpathlineto{\pgfpointorigin}}
}
\tikzset{%
  raise line/.style={%
    decoration = {single line, raise=#1},
    decorate},
  shorten risen line/.style args={shorten by #1 risen by #2}{%
    decoration = {%
      single line,
      raise = #2
   },
   decorate,
   shorten <=#1,
   shorten >=#1},
  label line/.style={%
    postaction = {%
      decoration = {raise=0pt,markings,mark=#1},
      decorate}},
  Voltage>/.style 2 args={%
    ->,
    thick,
    shorten risen line = shorten by 0.1cm risen by #1,
    label line = {%
      at position .5 with {\node[\ifdim#1>\z@ above\else
        below\fi,transform shape]{#2};}}},
  Voltage</.style 2 args={%
    <-,
    thick,
    shorten risen line = shorten by 0.1cm risen by #1,
    label line = {%
      at position .5 with {\node[\ifdim#1>\z@ above\else
        below\fi,transform shape]{#2};}}}
}

% Charge on a capacitor
\tikzset{%
  charge info generic/.style n args = {4}{% 
    append after command = {%
      \bgroup
        [current point is local = true]
        [every charge info/.try]
        [shift = {(\tikzlastnode.#1)}]
        node[#2,
             inner sep = 0pt,
             font = \footnotesize,
             #3] {#4}
      \egroup
    }
  },
  charge info north east/.style = {% 
    charge info generic = {north east}{right}{}{#1}},
  charge info north west/.style = {% 
    charge info generic = {north west}{left}{}{#1}},
  charge info south east/.style = {% 
    charge info generic = {south east}{right}{}{#1}},
  charge info south west/.style = {% 
    charge info generic = {south west}{left}{}{#1}},
  charge info sloped north east/.style = {% 
    charge info generic = {north east}{right}{transform shape}{#1}},
  charge info sloped north west/.style = {% 
    charge info generic = {north west}{left}{transform shape}{#1}},
  charge info sloped south east/.style = {% 
    charge info generic = {south east}{right}{transform shape}{#1}},
  charge info sloped south west/.style = {% 
    charge info generic = {south west}{left}{transform shape}{#1}},
}

\tikzset{%
  circuit declare symbol=dummy dipole,
  set dummy dipole graphic={%
    draw,
    shape=cloud,
    minimum width=1cm,
    minimum height=.5cm,
    transform shape}}
\pgfdeclareshape{ground IEC}
{
  \inheritsavedanchors[from=rectangle ee]
  \inheritanchor[from=rectangle ee]{center}
  \inheritanchor[from=rectangle ee]{north}
  \inheritanchor[from=rectangle ee]{south}
  \inheritanchor[from=rectangle ee]{east}
  \inheritanchor[from=rectangle ee]{west}
  \inheritanchor[from=rectangle ee]{north east}
  \inheritanchor[from=rectangle ee]{north west}
  \inheritanchor[from=rectangle ee]{south east}
  \inheritanchor[from=rectangle ee]{south west}
  \inheritanchor[from=rectangle ee]{input}
  \inheritanchor[from=rectangle ee]{output}
  \inheritanchorborder[from=rectangle ee]

  \backgroundpath{%
    \pgf@process{%
      \pgfpointadd{\southwest}{%
        \pgfpoint{\pgfkeysvalueof{/pgf/outer xsep}}%
        {\pgfkeysvalueof{/pgf/outer ysep}}}}
    \pgf@xa=\pgf@x
    \pgf@ya=\pgf@y
    \pgf@process{%
      \pgfpointadd{\northeast}{%
        \pgfpointscale{-1}{%
          \pgfpoint{\pgfkeysvalueof{/pgf/outer xsep}}%
          {\pgfkeysvalueof{/pgf/outer ysep}}}}}
    \pgf@xb=\pgf@x
    \pgf@yb=\pgf@y
    % On s'abaisse un peu
    \pgf@xc=.5\pgf@xb
    \pgf@yc=\pgf@ya
    \advance\pgf@yc\pgf@yb
    \pgfpathmoveto{\pgfqpoint{\pgf@xa}{\pgf@yc}}
    \pgfpathlineto{\pgfqpoint{\pgf@xc}{\pgf@yc}}
    % First line, simple
    \pgfpathmoveto{\pgfqpoint{\pgf@xc}{\pgf@ya}}
    \pgfpathlineto{\pgfqpoint{\pgf@xc}{\pgf@yb}}
    % Hachure
    \pgfmathsetlength\pgfutil@tempdima{(\pgf@yb-\pgf@ya)*.1}
    % \pgf@xa reste constant
    % \pgf@xb reste constant et égal à \pgf@xa + longueur du trait
    \pgfmathsetlength\pgf@xb{\pgf@xc+.1cm}
    \pgfmathloop\ifnum\pgfmathcounter<12\relax
      \pgfmathsetlength\pgf@yb{\pgf@ya+.5\pgfutil@tempdima}
      \pgfpathmoveto{\pgfqpoint{\pgf@xc}{\pgf@ya}}
      \pgfpathlineto{\pgfqpoint{\pgf@xb}{\pgf@yb}}
      \pgfmathsetlength\pgf@ya{\pgf@ya+\pgfutil@tempdima}
    \repeatpgfmathloop
  }
}
\pgfdeclareshape{var make contact IEC}
{
  \savedanchor\northeast{%
    \pgfmathsetlength\pgf@xa{\pgfkeysvalueof{/pgf/outer xsep}}%
    \pgfmathsetlength\pgf@xb{\pgfkeysvalueof{/pgf/minimum width}}%
    \pgf@x=\pgf@xa%
    \advance\pgf@x by .5\pgf@xb%
    \pgfmathsetlength\pgf@ya{\pgfkeysvalueof{/pgf/outer ysep}}%
    \pgfmathsetlength\pgf@yb{\pgfkeysvalueof{/pgf/minimum height}}%
    \pgf@y=\pgf@ya%
    \advance\pgf@y by\pgf@yb%
  }
  \savedanchor\southwest{%
    \pgfmathsetlength\pgf@xa{\pgfkeysvalueof{/pgf/outer xsep}}%
    \pgfmathsetlength\pgf@xb{\pgfkeysvalueof{/pgf/minimum width}}%
    \pgf@x=-.5\pgf@xa%
    \advance\pgf@x by -.5\pgf@xb%
    \pgfmathsetlength\pgf@ya{\pgfkeysvalueof{/pgf/outer ysep}}%
    \pgf@y=-\pgf@ya%
    \pgf@xc=0.083333\pgf@x%
    \advance\pgf@y by\pgf@xc%
  }

  \anchor{center}{\pgfpointorigin}
  \inheritanchor[from=rectangle ee]{north}
  \inheritanchor[from=rectangle ee]{south}
  \inheritanchor[from=rectangle ee]{east}
  \inheritanchor[from=rectangle ee]{west}
  \inheritanchor[from=rectangle ee]{north east}
  \inheritanchor[from=rectangle ee]{north west}
  \inheritanchor[from=rectangle ee]{south east}
  \inheritanchor[from=rectangle ee]{south west}
  \inheritanchor[from=rectangle ee]{input}
  \inheritanchor[from=rectangle ee]{output}

  \anchorborder{%
    \ifdim\pgf@y<0pt%
      % tricky... simpilfy to the origin...
      \pgf@xc=\pgf@x%
      \pgf@yc=\pgf@y%
      \pgf@process{\southwest}%
      \pgf@xa=\pgf@x%
      \pgf@ya=\pgf@y%
      \pgf@process{\pgfpointborderrectangle{\pgfqpoint{\pgf@xc}{\the\pgf@yc}}{\pgfqpoint{-\pgf@xa}{-\pgf@ya}}}%
    \else%
      \pgf@xc=\pgf@x%
      \pgf@yc=\pgf@y%
      \pgf@process{\pgfpointborderrectangle{\pgfqpoint{\pgf@xc}{\the\pgf@yc}}{\northeast}}%
    \fi%
  }

  \backgroundpath{
    \pgf@process{\pgfpointadd{\northeast}{
        \pgfpointscale{-1}{\pgfpoint{\pgfkeysvalueof{/pgf/outer xsep}}{\pgfkeysvalueof{/pgf/outer ysep}}}}}
    \pgf@xa=-\pgf@x \pgf@ya=0pt
    \pgf@xb=\pgf@x \pgf@yb=\pgf@y
    \pgf@xc=\pgf@xa
    \pgfutil@tempdima=2\pgf@xb%
    \pgfutil@tempdima=0.083333\pgfutil@tempdima%
    \advance\pgf@xa by \pgfutil@tempdima
    % Circle
    {\pgfpathcircle{\pgfqpoint{\pgf@xa}{0pt}}{\pgfutil@tempdima}}
    % Height
    % Start point
    \pgf@process{\pgfpointnormalised{\pgfpointdiff{\pgfqpoint{\pgf@xa}{0pt}}{\pgfqpoint{\pgf@xb}{\pgf@yb}}}}
    \pgf@xc=\pgf@x
    \pgf@yc=\pgf@y
    \pgfpathmoveto{\pgfpointadd{\pgfqpoint{\pgf@xa}{0pt}}{%
        \pgfpointscale{\pgfutil@tempdima}{\pgfqpoint{\pgf@xc}{\pgf@yc}}}}
    \pgfpathlineto{\pgfqpoint{\pgf@xb}{\pgf@yb}}
    % Circle
    \advance\pgf@xb by -\pgfutil@tempdima
    {\pgfpathcircle{\pgfqpoint{\pgf@xb}{\pgf@ya}}{\pgfutil@tempdima}}
  }
}
\pgfdeclareshape{var made contact IEC}
{
  \savedanchor\northeast{%
    \pgfmathsetlength\pgf@xa{\pgfkeysvalueof{/pgf/outer xsep}}%
    \pgfmathsetlength\pgf@xb{\pgfkeysvalueof{/pgf/minimum width}}%
    \pgf@x=\pgf@xa%
    \advance\pgf@x by .5\pgf@xb%
    \pgfmathsetlength\pgf@ya{\pgfkeysvalueof{/pgf/outer ysep}}%
    \pgfmathsetlength\pgf@yb{\pgfkeysvalueof{/pgf/minimum height}}%
    \pgf@y=\pgf@ya%
    \advance\pgf@y by\pgf@yb%
  }
  \savedanchor\southwest{%
    \pgfmathsetlength\pgf@xa{\pgfkeysvalueof{/pgf/outer xsep}}%
    \pgfmathsetlength\pgf@xb{\pgfkeysvalueof{/pgf/minimum width}}%
    \pgf@x=-.5\pgf@xa%
    \advance\pgf@x by -.5\pgf@xb%
    \pgfmathsetlength\pgf@ya{\pgfkeysvalueof{/pgf/outer ysep}}%
    \pgf@y=-\pgf@ya%
    \pgf@xc=0.083333\pgf@x%
    \advance\pgf@y by\pgf@xc%
  }

  \anchor{center}{\pgfpointorigin}
  \inheritanchor[from=rectangle ee]{north}
  \inheritanchor[from=rectangle ee]{south}
  \inheritanchor[from=rectangle ee]{east}
  \inheritanchor[from=rectangle ee]{west}
  \inheritanchor[from=rectangle ee]{north east}
  \inheritanchor[from=rectangle ee]{north west}
  \inheritanchor[from=rectangle ee]{south east}
  \inheritanchor[from=rectangle ee]{south west}
  \inheritanchor[from=rectangle ee]{input}
  \inheritanchor[from=rectangle ee]{output}

  \anchorborder{%
    \ifdim\pgf@y<0pt%
      % tricky... simpilfy to the origin...
      \pgf@xc=\pgf@x%
      \pgf@yc=\pgf@y%
      \pgf@process{\southwest}%
      \pgf@xa=\pgf@x%
      \pgf@ya=\pgf@y%
      \pgf@process{\pgfpointborderrectangle{\pgfqpoint{\pgf@xc}{\the\pgf@yc}}{\pgfqpoint{-\pgf@xa}{-\pgf@ya}}}%
    \else%
      \pgf@xc=\pgf@x%
      \pgf@yc=\pgf@y%
      \pgf@process{\pgfpointborderrectangle{\pgfqpoint{\pgf@xc}{\the\pgf@yc}}{\northeast}}%
    \fi%
  }

  \backgroundpath{
    \pgf@process{\pgfpointadd{\northeast}{
        \pgfpointscale{-1}{\pgfpoint{\pgfkeysvalueof{/pgf/outer xsep}}{\pgfkeysvalueof{/pgf/outer ysep}}}}}
    \pgf@xa=-\pgf@x \pgf@ya=0pt
    \pgf@xb=\pgf@x \pgf@yb=\pgf@y
    \pgf@xc=\pgf@xa
    \pgfutil@tempdima=2\pgf@xb%
    \pgfutil@tempdima=0.083333\pgfutil@tempdima%
    \advance\pgf@xa by \pgfutil@tempdima
    % Circle
    {\pgfpathcircle{\pgfqpoint{\pgf@xa}{0pt}}{\pgfutil@tempdima}}
    % Height
    % Start point
    \pgf@process{%
      \pgfpointnormalised{%
        \pgfpointdiff{\pgfqpoint{\pgf@xa}{0pt}}{\pgfqpoint{\pgf@xb}{\pgf@yb}}}}
    \pgf@xc=\pgf@x
    \pgf@yc=\pgf@y
    \pgfpathmoveto{\pgfpointadd{\pgfqpoint{\pgf@xa}{0pt}}{%
        \pgfpointscale{\pgfutil@tempdima}{\pgfqpoint{\pgf@xc}{\pgf@yc}}}}
    \pgfpathlineto{%
        \pgfpointadd{\pgfqpoint{\pgf@xb}{0pt}}{%
        \pgfpointscale{2\pgfutil@tempdima}{\pgfqpoint{-\pgf@xc}{\pgf@yc}}}}
    % Circle
    \advance\pgf@xb by -\pgfutil@tempdima
    {\pgfpathcircle{\pgfqpoint{\pgf@xb}{\pgf@ya}}{\pgfutil@tempdima}}
  }
}

\tikzset{%
  circuit declare symbol = ammeter,
  set ammeter graphic = {%
    draw,
    generic circle IEC,
    minimum size = 5mm}
  % every voltmeter/.append style={info={$A$}}
    }

\tikzset{%
  circuit declare symbol = voltmeter,
  set voltmeter graphic = {%
    draw,
    generic circle IEC,
    minimum size = 5mm},
  % every voltmeter/.append style={info={$V$}}
}

\tikzset{%
  circuit declare symbol = made contact,
  circuit ee IEC/.append style = {%
     set made contact graphic = var made contact IEC graphic},
  var made contact IEC graphic/.style = {
    circuit symbol wires,
    circuit symbol size = width 2 height 1,
    transform shape,
    shape = var made contact IEC,
    outer sep=0pt,
    cap=round
  },
}
\tikzset{%
  circuit ee IEC/.append style = {%
    set make contact graphic = var make contact IEC graphic},
}

\tikzset{circuit declare symbol = OA}
\tikzset{
  circuit ee IEC/.append style=
    {set OA graphic = OA IEC graphic}}
\tikzset{%
    OA IEC graphic/.style={
      circuit symbol open,
      circuit symbol size=width 6 height 4.5,
      shape=and gate IEC,
      and gate IEC symbol = {$\scriptstyle\triangleright\,\infty$},
      inner sep=.5ex
    }
  }
\tikzset{circuit declare symbol = MUL}
\tikzset{
  circuit ee IEC/.append style=
    {set MUL graphic = MUL IEC graphic}}
\tikzset{%
    MUL IEC graphic/.style={
      circuit symbol open,
      circuit symbol size=width 6 height 4.5,
      shape=and gate IEC,
      and gate IEC symbol = {$\times$},
      inner sep=.5ex
    }
  }
\tikzset{every ground/.style={point down,anchor=input}}

% Taken from circuitikz

\newdimen\pgf@circ@res@up \newdimen\pgf@circ@res@down \newdimen\pgf@circ@res@zero
\newdimen\pgf@circ@res@left \newdimen\pgf@circ@res@right
\newdimen\pgf@circ@res@other
\newdimen\pgf@circ@res@step

\def\circuitikzbasekey{/tikz/circuitikz}

\pgfkeys{\circuitikzbasekey/.is family}

\def\circuitikzset#1{\pgfkeys{\circuitikzbasekey,#1}}
\let\ctikzset\circuitikzset
\def\ctikzvalof#1{\pgfkeysvalueof{\circuitikzbasekey/#1}}
\def\ctikzsetvalof#1#2{\pgfkeyssetvalue{\circuitikzbasekey/#1}{#2}}

\ctikzset{bipoles/.is family}
\ctikzset{bipoles/border margin/.initial=1.1}
\ctikzset{bipoles/thickness/.initial=2} 
\ctikzset{bipoles/length/.initial=1.4cm} 
\ctikzset{nodes width/.initial=.04}
\newdimen\pgf@circ@Rlen 
\ctikzset{bipoles/length/.code={\pgf@circ@Rlen = #1}} 

\ctikzset{tripoles/spdt/width/.initial=.85}
\ctikzset{tripoles/spdt/height/.initial=.45}
\ctikzset{tripoles/spdt/margin/.initial=.45}

\ctikzset{thickness/.initial=2}
\ctikzset{color/.initial=black}

\pgfdeclareshape{ocirc}{
	\anchor{center}{
		\pgfpointorigin
	}
	\anchorborder{
		\pgf@circ@res@left=\pgf@x
		\pgf@circ@res@up=\pgf@y
		\pgfpointborderellipse{\pgfpoint{\pgf@circ@res@left}{\pgf@circ@res@up}
}{\pgfpoint{\pgfkeysvalueof{/tikz/circuitikz/nodes width}*\pgfkeysvalueof{/tikz/circuitikz/bipoles/length}}{\pgfkeysvalueof{/tikz/circuitikz/nodes width}*\pgfkeysvalueof{/tikz/circuitikz/bipoles/length}}}		
	}

	\behindforegroundpath{		
		
		\pgfscope
			\pgfpathcircle{\pgfpointorigin}{\pgfkeysvalueof{/tikz/circuitikz/nodes width}*\pgfkeysvalueof{/tikz/circuitikz/bipoles/length}}
			\pgfsetcolor{\pgfkeysvalueof{/tikz/circuitikz/color}}
			\pgfsetfillcolor{white}
			\pgfusepath{draw,fill}		
		\endpgfscope

		}
}


	\pgfdeclareshape{spdt}
	{
	  \savedanchor\northwest{%
		\pgf@y= \pgfkeysvalueof{/tikz/circuitikz/bipoles/length}
		\pgf@y=\pgfkeysvalueof{/tikz/circuitikz/tripoles/spdt/height}\pgf@y
		\pgf@y=.5\pgf@y
		\pgf@x= \pgfkeysvalueof{/tikz/circuitikz/bipoles/length}
		\pgf@x=-\pgfkeysvalueof{/tikz/circuitikz/tripoles/spdt/width}\pgf@x
		\pgf@x=.5\pgf@x
	  }
	  \anchor{left}{%
	  	\northwest
	  	\pgf@y=0pt
	  }
	  \anchor{in}{
	  	\northwest
	  	\pgf@y=0pt
	  }	  
	  \anchor{out 1}{
		\northwest
		\pgf@x=-\pgf@x
	  }
	  \anchor{out 2}{
		\northwest
		\pgf@x=-\pgf@x
		\pgf@y=-\pgf@y
	  }
  	  \anchor{center}{
		\pgf@y=0pt
		\pgf@x=0pt
	  }
	  \anchor{east}{
	  	\northwest
		\pgf@y=0pt
	  	\pgf@x=-\pgf@x  
	  }
	  \anchor{west}{
	  	\northwest
		\pgf@y=0pt
	  }
	  \anchor{south}{
		\northwest
		\pgf@x=0pt
		\pgf@y=-\pgf@y
	  }
	  \anchor{north}{
		\northwest
		\pgf@x=0pt
	  }
	  \anchor{south west}{
		\northwest
		\pgf@y=-\pgf@y
	  }
	  \anchor{north east}{
		\northwest
		\pgf@x=-\pgf@x
	  }
	  \anchor{north west}{
		\northwest
	  }
	  \anchor{south east}{
		\northwest
		\pgf@x=-\pgf@x
		\pgf@y=-\pgf@y
	  }	  
	  \backgroundpath{			
			\pgfsetcolor{\pgfkeysvalueof{/tikz/circuitikz/color}}	
			
			
			\northwest
			\pgf@circ@res@up = \pgf@y 
			\pgf@circ@res@down = -\pgf@y
			\pgf@circ@res@right = -\pgf@x  
			\pgf@circ@res@left = \pgf@x  
			\pgf@circ@res@other = \pgfkeysvalueof{/tikz/circuitikz/tripoles/spdt/margin}\pgf@circ@res@left
			
		
	  	\pgfpathmoveto{\pgfpoint{\pgf@circ@res@right}{\pgf@circ@res@up}}
		\pgfpathlineto{\pgfpoint{-\pgf@circ@res@other}{\pgf@circ@res@up}}
	  	\pgfpathmoveto{\pgfpoint{\pgf@circ@res@right}{\pgf@circ@res@down}}
		\pgfpathlineto{\pgfpoint{-\pgf@circ@res@other}{\pgf@circ@res@down}}
		
		\pgfpathmoveto{\pgfpoint{\pgf@circ@res@left}{0pt}}
		\pgfpathlineto{\pgfpoint{\pgf@circ@res@other}{0pt}}
		
		\pgfusepath{draw}
		
		\pgfscope
			\pgftransformshift{\pgfpoint{-\pgf@circ@res@other}{\pgf@circ@res@up}}
			\pgfnode{ocirc}{center}{}{spdt1}{\pgfusepath{stroke}}
		\endpgfscope
		\pgfscope
			\pgftransformshift{\pgfpoint{-\pgf@circ@res@other}{\pgf@circ@res@down}}
			\pgfnode{ocirc}{center}{}{}{\pgfusepath{stroke}}
		\endpgfscope
		\pgfscope
			\pgftransformshift{\pgfpoint{\pgf@circ@res@other}{0pt}}
			\pgfnode{ocirc}{center}{}{spdt2}{\pgfusepath{stroke}}
		\endpgfscope
		
		
		\pgfscope
			\pgfpathmoveto{\pgfpointshapeborder{spdt2}{\pgfpointorigin}}
			\pgfpathlineto{
				\pgfpointadd{\pgfpointshapeborder{spdt1}{\pgfpoint{-\pgf@circ@res@other}{-100pt}}}
				{\pgfpoint{-.05\pgf@circ@res@up}{-.05\pgf@circ@res@up}}
			}
			\pgfsetlinewidth{2\pgflinewidth}
			\pgfusepath{draw}
		\endpgfscope
	  }
	}

\makeatother
% Local Variables:
% coding: utf-8
% End:
